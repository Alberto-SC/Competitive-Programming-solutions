\documentclass[11pt]{article}
\usepackage[utf8]{inputenc}
\usepackage[spanish]{babel}
\usepackage{amsmath}
\usepackage{amsthm}  
\usepackage{amssymb}
\usepackage{graphicx}
\usepackage[margin=0.8in,landscape]{geometry}
\usepackage{fancyhdr}
\usepackage[inline]{enumitem}
\usepackage{float}
\usepackage{cancel}
\usepackage{bigints}
\usepackage{listings}
\usepackage{xcolor}
\usepackage{listingsutf8}
\usepackage{algpseudocode}
\usepackage{algorithm}
\usepackage{apacite}
\usepackage{multicol}
\usepackage{hyperref}
\usepackage{titlesec}
\usepackage{xparse}
\usepackage{geometry}
\usepackage{mathtools}
\DeclarePairedDelimiter\ceil{\lceil}{\rceil}
\DeclarePairedDelimiter\floor{\lfloor}{\rfloor}
\newcommand\TestAppExists[3]{#2}
\usepackage{minted}
\DeclarePairedDelimiter\ceil{\lceil}{\rceil}
\DeclarePairedDelimiter\floor{\lfloor}{\rfloor}
 \geometry{
 a4paper,
 total={170mm,257mm},
 left=20mm,
 top=20mm,
 }
\ExplSyntaxOn
\NewDocumentCommand{\mintedpath}{m}
 {
  \seq_gset_split:Nnn \g_paulie_mintedpath_seq { } { #1 }
  \seq_gput_left:Nn \g_paulie_mintedpath_seq { }
 }

\seq_new:N \g_paulie_mintedpath_seq

\NewDocumentCommand{\pathinputminted}{O{}mm}
 {
  \seq_map_inline:Nn \g_paulie_mintedpath_seq
   {
    \file_if_exist:nT { ##1 #3 }
     {
      \inputminted[#1]{#2}{##1 #3}
      \seq_map_break:
     }
   }
 }
\ExplSyntaxOff
\newtheorem*{remark}{Remark}
\mintedpath{ {Number_theory/}  }
\setcounter{secnumdepth}{4}

% \twocolumn
\titleformat{\paragraph}
{\normalfont\normalsize\bfseries}{\theparagraph}{1em}{}
\titlespacing*{\paragraph}
{0pt}{3.25ex plus 1ex minus .2ex}{1.5ex plus .2ex}
\decimalpoint
\hypersetup{
	colorlinks,
	citecolor=black,
	filecolor=black,
	linkcolor=black,
	urlcolor=black
}
\usepackage{tipa}
\pagestyle{fancy}
\newcommand{\xvdash}[1]{%
\vdash^{\mkern-10mu\scriptscriptstyle\rule[-.9ex]{0pt}{0pt}#1}%
}
\setlength{\headheight}{15pt} 
\rhead{\thepage}

\lfoot{ESCOM-IPN}
\renewcommand{\footrulewidth}{0.5pt
\setlength{\parskip}{0.5em}
\newcommand{\ve}[1]{\overrightarrow{#1}}
\newcommand{\abs}[1]{\left\lvert #1 \right\lvert}
\newcommand{\blank}{\text{\textcrb}}
% \title{Reference}

\lstdefinestyle{customc}{
	belowcaptionskip=1\baselineskip,
	breaklines=true,
	frame=L,
	xleftmargin=\parindent,
	language=C++,
	showstringspaces=false,
	basicstyle=\ttfamily,
	keywordstyle=\bfseries\color{green!40!black},
	commentstyle=\itshape\color{purple!40!black},
	identifierstyle=\color{blue},
	numbers=left,
	stringstyle=\color{orange},
}
	
 
	\setlength{\columnseprule}{1pt}
	% \def\columnseprulecolor{\color{Gray}}	
	\begin{center}
		\Huge\textsc{ICPC Reference}\\
		\vspace{0.30cm}
		\huge Escuela Superior de Cómputo - IPN\\
		\vspace{0.20cm}
		\small Alberto Silva
	\end{center}
	\hrulefill
	\begin{multicols}{3}
	\tableofcontents	
	\clearpage
	\section{Matematicas}
	\subsection{Fundamentals}
			% \subsection{Facts}
			% \[ \sum_{i=1}^{\infty} \frac{1}{n^s} = \prod_p \frac{1}{1 - p^{-s}} \]
			% \begin{align*}
			% 	x&=y           &  w &=z              &  a&=b+c\\
			% 	2x&=-y         &  3w&=\frac{1}{2}z   &  a&=b\\
			% 	-4 + 5x&=2+y   &  w+2&=-1+w          &  ab&=cb
			% \end{align*}
			% \begin{gather*} 
			% 	2x - 5y =  8 \\ 
			% 	3x^2 + 9y =  3a + c
			% \end{gather*}
			% \begin{align*} 
			% 	2x - 5y &=  8 \\ 
			% 	3x + 9y &=  -12
			% 	Prime numbers:\\\
			% 	998244353;
			% \end{align*}
			\subsubsection{Exponenciación y multiplicación binaria}
			\pathinputminted[tabsize=2,breaklines,firstline=36,lastline=56,fontsize=\small]{c++}{Number_theory.cpp}
			% \large Aplications
			% % \begin{itemize}
			% % 	\item One entry in the list
			% % 	\item Another entry in the list
			% % \end{itemize}
			% \newpage


			\subsubsection{Mínimo común multiplo y máximo cómun divisor}
			\pathinputminted[tabsize=2,breaklines,firstline=140,lastline=152,fontsize=\small]{c++}{Number_theory.cpp}
 
			\subsubsection{Euclides extendido}
			\pathinputminted[tabsize=2,breaklines,firstline=77,lastline=82,fontsize=\small]{c++}{Number_theory.cpp}
	
			% \subsubsection{Linear Diophantine Equation}
			% \pathinputminted[tabsize=2,breaklines,firstline=112, lastline=122,fontsize=\small]{c++}{Number_theory.cpp}
			
	\newpage
	\subsection{Aritmetica modular}	
			\subsubsection{Inverso modular}
			\pathinputminted[tabsize=2,breaklines,firstline=84,lastline=88,fontsize=\small]{c++}{Number_theory.cpp}						
			\pathinputminted[tabsize=2,breaklines,firstline=92,lastline=100,fontsize=\small]{c++}{Number_theory.cpp}						
			
			\subsubsection{Linear Congruence Equation}
			\pathinputminted[tabsize=2,breaklines,firstline=112,lastline=122,fontsize=\small]{c++}{Number_theory.cpp}						
			
			\subsubsection{Factorial modulo p}
			\pathinputminted[tabsize=2,breaklines,firstline=161,lastline=170  ,fontsize=\small]{c++}{Number_theory.cpp}						

			\subsubsection{Chinese Remainder Theorem}
			\pathinputminted[tabsize=2,breaklines,firstline=177,lastline=192,fontsize=\small]{c++}{Number_theory.cpp}

% 			\subsubsection{Discrete Root}
% 			The problem of finding a discrete root is defined as follows. 
% 			Given a prime n and two integers a and k, find all x for which:\\
% 			$$x^{k} \enskip (mod \enskip m)$$ 
		
% \subsubsection{Primitive Root}

% 			\subsubsection{Discrete Logarithm}

% 			\subsubsection{Montgomery Multiplication}

	\subsection{Cribas y primos}
			\subsubsection{Criba de eratostenes}
			\pathinputminted[tabsize=2,breaklines,firstline=26,lastline=36,fontsize=\small]{c++}{Primes.cpp}
			
			\subsubsection{Criba de factor primo más pequeño}
			\pathinputminted[tabsize=2,breaklines,firstline=55,lastline=66,fontsize=\small]{c++}{Primes.cpp}
						
			\subsubsection{Criba de la función $\varphi$ de Euler}
			\pathinputminted[tabsize=2,breaklines,firstline=79,lastline=88,fontsize=\small]{c++}{Primes.cpp}
			
			\subsubsection{Criba de la función $\mu$}
			\pathinputminted[tabsize=2,breaklines,firstline=90,lastline=98,fontsize=\small]{c++}{Primes.cpp}
			
			\subsubsection{Triángulo de Pascal}
			\pathinputminted[tabsize=2,breaklines,firstline=100,lastline=110,fontsize=\small]{c++}{Primes.cpp}
			
			% \subsubsection{Segmented sieve}
			% \inputminted[tabsize=2,breaklines,firstline=860,lastline=892,fontsize=\small]{c++}{numberTheory.cpp}
			
			\subsubsection{Criba de primos lineal}
			\pathinputminted[tabsize=2,breaklines,firstline=41,lastline=53,fontsize=\small]{c++}{Primes.cpp}
			
			% \  subsubsection{Criba lineal para funciones multiplicativas}
			% \inputminted[tabsize=2,breaklines,firstline=827,lastline=858,fontsize=\small]{c++}{numberTheory.cpp}
			
			\subsubsection{Block sieve}
			\pathinputminted[tabsize=2,breaklines,firstline=126,lastline=157,fontsize=\small]{c++}{Primes.cpp}

			\subsubsection{Prime factors of $n!$}
			if $p$ is prime the highest power $p^{k}$ of $p$ that divides $n!$ is given by 
			\begin{align*}
				k = \floor*{\frac{n}{p}}  + \floor*{\frac{n}{p^{2}}} +
				 \floor*{\frac{n}{p^{3}}} + \cdots	
			\end{align*}

			\subsubsection{Primaly test(miller rabin)}
			\pathinputminted[tabsize=2,breaklines,firstline=184,lastline=232,fontsize=\small]{c++}{Primes.cpp}
			
			% \subsubsection{Potencia de un primo que divide a un factorial}
			% \pathinputminted[tabsize=2,breaklines,firstline=99,lastline=109,fontsize=\small]{c++}{Primes.cpp}

			% \subsubsection{Factorización de un número}
			% \pathinputminted[tabsize=2,breaklines,firstline=99,lastline=109,fontsize=\small]{c++}{Primes.cpp}

			% \subsubsection{Factorización de un factorial}
			% \pathinputminted[tabsize=2,breaklines,firstline=99,lastline=109,fontsize=\small]{c++}{Primes.cpp}

			\subsubsection{Factorización varios metodos}
			\pathinputminted[tabsize=2,breaklines,firstline=302,lastline=485,fontsize=\small]{c++}{Primes.cpp}

			\subsubsection{Factorizacion usando todos los metodos}
			\pathinputminted[tabsize=2,breaklines,firstline=486,lastline=514,fontsize=\small]{c++}{Primes.cpp}

			\subsubsection{Numero de divisores hasta 1018}
			\pathinputminted[tabsize=2,breaklines,firstline=268,lastline=299,fontsize=\small]{c++}{Primes.cpp}
		
			\subsection{Funciones multiplicativas}
			\subsubsection{Función $\varphi$ de Euler}
			\pathinputminted[tabsize=2,breaklines,firstline=123,lastline=135,fontsize=\small]{c++}{Number_theory.cpp}
			The most famous and important property of Euler's totient function
			 is expressed in \textbf{Euler's theorem:}
			\\
			\begin{align}
				\alpha^{\phi(m)} \equiv 1 (mod  \quad m)
			\end{align}
			if \textbf{\textit{$\alpha$}}
			and \textbf{\textit{m}} 
			are relative prime.\\
			In the particular case when m is prime,
			Euler's theorem turns into \textbf{Fermat's little theorem:}
			\begin{align}
				\alpha^{m-1}\equiv 1 (mod  \quad m)
			\end{align}
			\smallskip
			\begin{align}
				\alpha^{n}\equiv \alpha^{n \enskip mod \enskip\phi(m)} \quad (mod  \quad m)
			\end{align}
			This allows computing $x^{n} mod \quad m$ for very big $n$, especially if
			n is the result of another computation, 
			as it allows to compute n under a modulo.
			% \newpage
			% \subsubsection{Función $\varphi^-1$ de Euler}
				% \subsubsection{Función $\sigma$}
				% % \inputminted[tabsize=2,breaklines,firstline=209,lastline=226,fontsize=\small]{c++}{numberTheory.cpp}
				
				% \subsubsection{Función $\Omega$}
				% % \inputminted[tabsize=2,breaklines,firstline=228,lastline=235,fontsize=\small]{c++}{numberTheory.cpp}
				
				% \subsubsection{Función $\omega$}
				% % \inputminted[tabsize=2,breaklines,firstline=237,lastline=244,fontsize=\small]{c++}{numberTheory.cpp}
								
				% \subsubsection{Función $\mu$}
				% % \inputminted[tabsize=2,breaklines,firstline=276,lastline=287,fontsize=\small]{c++}{numberTheory.cpp}
				
				% \subsubsection{Number of divisors / sum of divisors}
			

		% \subsection{Orden multiplicativo, raíces primitivas y raíces de la unidad}
		% 	\subsubsection{Función $\lambda$ de Carmichael}
		% 	% \inputminted[tabsize=2,breaklines,firstline=260,lastline=274,fontsize=\small]{c++}{numberTheory.cpp}
			
		% 	\subsubsection{Orden multiplicativo módulo $m$}
		% 	% \inputminted[tabsize=2,breaklines,firstline=289,lastline=305,fontsize=\small]{c++}{numberTheory.cpp}
			
		% 	\subsubsection{Número de raíces primitivas (generadores) módulo $m$}
		% 	% \inputminted[tabsize=2,breaklines,firstline=307,lastline=313,fontsize=\small]{c++}{numberTheory.cpp}
			
		% 	\subsubsection{Test individual de raíz primitiva módulo $m$}
		% 	% \inputminted[tabsize=2,breaklines,firstline=315,lastline=325,fontsize=\small]{c++}{numberTheory.cpp}
			
		% 	\subsubsection{Test individual de raíz $k$-ésima de la unidad módulo $m$}
		% 	% \inputminted[tabsize=2,breaklines,firstline=327,lastline=336,fontsize=\small]{c++}{numberTheory.cpp}
			
		% 	\subsubsection{Encontrar la primera raíz primitiva módulo $m$}
		% 	% \inputminted[tabsize=2,breaklines,firstline=338,lastline=355,fontsize=\small]{c++}{numberTheory.cpp}
			
		% 	\subsubsection{Encontrar la primera raíz $k$-ésima de la unidad módulo $m$}
		% 	% \inputminted[tabsize=2,breaklines,firstline=357,lastline=373,fontsize=\small]{c++}{numberTheory.cpp}
			
		% 	\subsubsection{Logaritmo discreto}
		% 	% \inputminted[tabsize=2,breaklines,firstline=375,lastline=398,fontsize=\small]{c++}{numberTheory.cpp}
			
		% 	\subsubsection{Raíz $k$-ésima discreta}
		% 	% \inputminted[tabsize=2,breaklines,firstline=400,lastline=416,fontsize=\small]{c++}{numberTheory.cpp}
			
	% 	\subsection{Particiones}
	% 		\subsubsection{Función $P$ (particiones de un entero positivo)}
	% 		% \inputminted[tabsize=2,breaklines,firstline=519,lastline=547,fontsize=\small]{c++}{numberTheory.cpp}
			
	% 		\subsubsection{Función $Q$ (particiones de un entero positivo en distintos sumandos)}
	% 		% \inputminted[tabsize=2,breaklines,firstline=549,lastline=596,fontsize=\small]{c++}{numberTheory.cpp}
			
	% 		\subsubsection{Número de factorizaciones ordenadas}
	% 		% \inputminted[tabsize=2,breaklines,firstline=743,lastline=771,fontsize=\small]{c++}{numberTheory.cpp}
			
	% 		\subsubsection{Número de factorizaciones no ordenadas}
	% 		% \inputminted[tabsize=2,breaklines,firstline=773,lastline=799,fontsize=\small]{c++}{numberTheory.cpp}
			
	% \newpage
		% \subsection{Números racionales}
		% 	\subsubsection{Estructura \texttt{fraccion}}
		% 	% \inputminted[tabsize=2,breaklines,firstline=7,lastline=123,fontsize=\small]{c++}{fraccion.cpp}
		
		% \new+page
		\subsection{Linear Algebra}
			\subsubsection{Struct matrix} 
			\pathinputminted[tabsize=2,breaklines,firstline=6,lastline=134,fontsize=\small]{c++}{Algebra_lineal.cpp}
			\pathinputminted[tabsize=2,breaklines,firstline=335,lastline=335,fontsize=\small]{c++}{Algebra_lineal.cpp}
			
			\subsubsection{Transpuesta}
			\pathinputminted[tabsize=2,breaklines,firstline=212,lastline=220,fontsize=\small]{c++}{Algebra_lineal.cpp}
			
			\subsubsection{Traza}
			\pathinputminted[tabsize=2,breaklines,firstline=222,lastline=227,fontsize=\small]{c++}{Algebra_lineal.cpp}
						
			\subsubsection{Gauss  System of Linear Equationsn}
			\pathinputminted[tabsize=2,breaklines,firstline=288,lastline=332,fontsize=\small]{c++}{Algebra_lineal.cpp}
			
			\subsubsection{Gauss  Determinant}
			\pathinputminted[tabsize=2,breaklines,firstline=255,lastline=286,fontsize=\small]{c++}{Algebra_lineal.cpp}
			
			\subsubsection{Cofactors Matrix}
			\pathinputminted[tabsize=2,breaklines,firstline=175,lastline=193,fontsize=\small]{c++}{Algebra_lineal.cpp}

			\subsubsection{Matriz inversa}
			\pathinputminted[tabsize=2,breaklines,firstline=136,lastline=173,fontsize=\small]{c++}{Algebra_lineal.cpp}
			
			\subsubsection{Adjoint Matrix}
			\pathinputminted[tabsize=2,breaklines,firstline=194,lastline=209,fontsize=\small]{c++}{Algebra_lineal.cpp}
			
			\subsubsection{Recurrencias lineales}
			\pathinputminted[tabsize=2,breaklines,firstline=351,lastline=370,fontsize=\small]{c++}{Algebra_lineal.cpp}
			
			\subsubsection{Kirchhoff Matrix Tree Theorem}
			Count the number of spanning trees in a graph, as the determinant of the Laplacian matrix of the graph. 
			\\
			\textbf{Laplacian Matrix} :
			\\Given a simple graph $G$ with $n$ vertices,
			its Laplacian matrix $L_{n\times n}$ is defined as\\
			\begin{align*}
				L=D-A
			\end{align*}
			The elements of $L$ are given by
			\[ L_{i,j} =
			\begin{cases}
				deg(v_{i})       & \quad \text{if } i == j\\
				-1  & \quad \text{if } i \neq j \text{and } v_{i} \text{ is adjacent to }v_{j}\\
				0 & \quad \text{otherwise}
			\end{cases}
			\]
			 define $\tau(G  )$ as number of spanning trees of a grap $G$
			\begin{align*}
				\tau(G) = \det L_{n-1\times n-1}\\
			\end{align*}
			Where  $L_{n-1 \times n-1}$ is a laplacian matrix deleting 
			any row and any column
			\begin{align*}
				\det
				\begin{pmatrix}
					deg(v_{1}) & L_{1,2} & \cdots & L_{1,n-1} \\
				L_{2,1} & deg(v_{2}) & \cdots & L_{2,n-1} \\
				\vdots  & \vdots  & \ddots & \vdots  \\
				L_{n-1,1} & L_{n-1,2} & \cdots & deg(v_{n-1}) 
				\end{pmatrix}
			\end{align*}
			Generalization for a multigraph $K_{n}^{m} \pm G$\\
			define $\tau(K_{n}^{m} \pm G)$ as number of spanning trees of a grap $K_{n}^{m} \pm G$
			\begin{align*}
				\tau(K_{n}^{m} \pm G) = n * (nm)^{n-p-2}\det (B)
			\end{align*}
			where $B = mnI_{p} +\alpha * L(G)$is a $p\times p$ matrix, $\alpha = \pm$  according 
			$(K_{n}^{m} \pm G)$, and $L(G)$ is the Kirchhoff	matrix of G
			\pathinputminted[tabsize=2,breaklines,firstline=447,lastline=465,fontsize=\small]{c++}{Algebra_lineal.cpp}
			% $
			% \left\{ \begin{array}{c} 3x-y+z=0 \\ x+2y-z=1\\-x+3y-z=-2 \end{array}\right 	
			% $
			% its Laplacian matrix {\textstyle L_{n\times n}}{\textstyle L_{n\times n}} is defined as:[1]
			% {\displaystyle L=D-A,}{\displaystyle L=D-A,}

			% \subsubsection{Linear Recurrence and Berlekamp-Massey Algorithm}
			% \inputminted[tabsize=2,breaklines,firstline=7,lastline=38,fontsize=\small]{c++}{recurrence.cpp}

			% \inputminted[tabsize=2,breaklines,firstline=341,lastline=376,fontsize=\small]{c++}{matrix.cpp}
			
			% \inputminted[tabsize=2,breaklines,firstline=378,lastline=394,fontsize=\small]{c++}{matrix.cpp}
			
			% \subsubsection{Polinomio característico}
			% \subsubsection{Rank of a matrix}
			% \inputminted[tabsize=2,breaklines,firstline=396,lastline=406,fontsize=\small]{c++}{matrix.cpp}
			
			% \subsubsection{Gram-Schmidt}
			% \inputminted[tabsize=2,breaklines,firstline=408,lastline=422,fontsize=\small]{c++}{matrix.cpp}
			

		\subsection{Metodos numericos}
			\subsubsection{FFT}
			\pathinputminted[tabsize=2,breaklines,firstline=5,lastline=84,fontsize=\small]{c++}{numeric_methods.cpp}
				
				% \subsubsection{FFT con raíces de la unidad complejas}
				% % \inputminted[tabsize=2,breaklines,firstline=13,lastline=44,fontsize=\small]{c++}{fft.cpp}
				
				% \subsubsection{FFT con raíces de la unidad discretas (NTT)}
				% % \inputminted[tabsize=2,breaklines,firstline=46,lastline=90,fontsize=\small]{c++}{fft.cpp}
				% 	\subsubsection{Otros valores para escoger la raíz y el módulo}
				% 		\begin{table}[H]
				% 			\centering
				% 			\begin{tabular}{|p{2cm}|p{1.7cm}|p{2cm}|p{4.5cm}|}
				% 				\hline
				% 				Raíz $n$-ésima de la unidad ($\omega$) & $\omega^{-1}$ & Tamaño máximo del arreglo ($n$) & Módulo $p$ \\ \hline
				% 				15 & 30584 & $2^{14}$ & $4 \times 2^{14} + 1 = 65537$ \\ \hline
				% 				9 & 7282 & $2^{15}$ & $2 \times 2^{15} + 1 = 65537$ \\ \hline
				% 				3 & 21846 & $2^{16}$ & $1 \times 2^{16} + 1 = 65537$ \\ \hline
				% 				8 & 688129 & $2^{17}$ & $6 \times 2^{17} + 1 = 786433$ \\ \hline
				% 				5 & 471860 & $2^{18}$ & $3 \times 2^{18} + 1 = 786433$ \\ \hline
				% 				12 & 3364182 & $2^{19}$ & $11 \times 2^{19} + 1 = 5767169$ \\ \hline
				% 				\textbf{5} & \textbf{4404020} & $\mathbf{2^{20}}$ & $7 \times 2^{20} + 1 = \textbf{7340033}$ \\ \hline
				% 				38 & 21247462 & $2^{21}$ & $11 \times 2^{21} + 1 = 23068673$ \\ \hline
				% 				21 & 49932191 & $2^{22}$ & $25 \times 2^{22} + 1 = 104857601$ \\ \hline
				% 				4 & 125829121 & $2^{23}$ & $20 \times 2^{23} + 1 = 167772161$ \\ \hline
				% 				\textbf{31} & \textbf{128805723} & $\mathbf{2^{23}}$ & $119 \times 2^{23} + 1 = \textbf{998244353}$ \\ \hline
				% 				2 & 83886081 & $2^{24}$ & $10 \times 2^{24} + 1 = 167772161$ \\ \hline
				% 				17 & 29606852 & $2^{25}$ & $5 \times 2^{25} + 1 = 167772161$ \\ \hline
				% 				30 & 15658735 & $2^{26}$ & $7 \times 2^{26} + 1 = 469762049$ \\ \hline
				% 				137 & 749463956 & $2^{27}$ & $15 \times 2^{27} + 1 = 2013265921$ \\ \hline
				% 			\end{tabular}
				% 		\end{table}
				
				
			% \subsubsection{Aplicaciones}
			% 	\subsubsection{Multiplicación de polinomios}
			% 	\pathinputminted[tabsize=2,breaklines,firstline=72,lastline=84,fontsize=\small]{c++}{numeric_methods.cpp}
				
			% 	\subsubsection{Multiplicación de números enteros grandes}
			% 	% \inputminted[tabsize=2,breaklines,firstline=120,lastline=156,fontsize=\small]{c++}{fft.cpp}
				
			% 	\subsubsection{Inverso de un polinomio}
			% 	% \inputminted[tabsize=2,breaklines,firstline=158,lastline=184,fontsize=\small]{c++}{fft.cpp}
				
			% 	\subsubsection{Raíz cuadrada de un polinomio}
			% 	% \inputminted[tabsize=2,breaklines,firstline=186,lastline=208,fontsize=\small]{c++}{fft.cpp}
			
			
				% \subsubsection{A Bitwise Convolution}
				% \subsubsection{Möbius inversion}
				% \subsubsection{Dirichlet convolution}
				% \subsubsection{Schonhage-Strassen}
				% \subsubsection{Integration by Simpson's formula}
				% \subsubsection{Newton's method for finding roots}
				% \subsubsection{Ternary Search}
			
			\subsection{Combinatorics}
				
				\subsubsection{Binomial coefficents}
				\pathinputminted[tabsize=2,breaklines,firstline=29,lastline=144,fontsize=\small]{c++}{Combinatorics.cpp}
				Distribute N items in m container HOLA
				\binom{N+m-1}{N}
			% \subsection{Otros}
			% 	\subsubsection{Fibonacci}
			% 	% \inputminted[tabsize=2,breaklines,firstline=720,lastline=741,fontsize=\small]{c++}{numberTheory.cpp}
		
			% 	\subsubsection{Cambio de base}
			% 	% \inputminted[tabsize=2,breaklines,firstline=442,lastline=462,fontsize=\small]{c++}{numberTheory.cpp}
				
			% 	\subsubsection{Fracciones continuas}
			% 	% \inputminted[tabsize=2,breaklines,firstline=598,lastline=640,fontsize=\small]{c++}{numberTheory.cpp}
				
			% 	\subsubsection{Ecuación de Pell}
			% 	% \inputminted[tabsize=2,breaklines,firstline=642,lastline=655,fontsize=\small]{c++}{numberTheory.cpp}
				
			% 	\subsubsection{Números de Bell}
			% 	% \inputminted[tabsize=2,breaklines,firstline=898,lastline=910,fontsize=\small]{c++}{numberTheory.cpp}
					

	\newpage
	% \section{Geometría}
	% 	\subsection{Estructura \texttt{point}}
	% % 	\inputminted[tabsize=2,breaklines,firstline=4,lastline=99,fontsize=\small]{c++}{geometry.cpp}
		
	% 	\subsection{Líneas y segmentos}
	% 		\subsusubbsection{Verificar si un punto pertenece a una línea o segmento}
	% 		\inputminted[tabsize=2,breaklines,firstline=102,lastline=111,fontsize=\small]{c++}{geometry.cpp}
			
	% 		\subsusubbsection{Intersección de líneas}
	% 		\inputminted[tabsize=2,breaklines,firstline=113,lastline=133,fontsize=\small]{c++}{geometry.cpp}
			
	% 		\subsubsection{Intersección línea-segmento}
	% 		\inputminted[tabsize=2,breaklines,firstline=135,lastline=148,fontsize=\small]{c++}{geometry.cpp}
			
	% 		\subsubsection{Intersección de segmentos}
	% 		\inputminted[tabsize=2,breaklines,firstline=150,lastline=167,fontsize=\small]{c++}{geometry.cpp}
			
	% 		\subsubsection{Distancia punto-recta}
	% 		\inputminted[tabsize=2,breaklines,firstline=169,lastline=172,fontsize=\small]{c++}{geometry.cpp}
			
	% 	\subsection{Círculos}
	% 		\subsubsection{Distancia punto-círculo}
	% 		\inputminted[tabsize=2,breaklines,firstline=391,lastline=394,fontsize=\small]{c++}{geometry.cpp}
			
	% 		\subsubsection{Proyección punto exterior a círculo}
	% 		\inputminted[tabsize=2,breaklines,firstline=396,lastline=399,fontsize=\small]{c++}{geometry.cpp}
			
	% 		\subsubsection{Puntos de tangencia de punto exterior}
	% 		\inputminted[tabsize=2,breaklines,firstline=401,lastline=406,fontsize=\small]{c++}{geometry.cpp}
			
	% 		\subsubsection{Intersección línea-círculo}
	% 		\inputminted[tabsize=2,breaklines,firstline=408,lastline=422,fontsize=\small]{c++}{geometry.cpp}
			
	% 		\subsubsection{Centro y radio a través de tres puntos}
	% 		\inputminted[tabsize=2,breaklines,firstline=424,lastline=429,fontsize=\small]{c++}{geometry.cpp}
			
	% 		\subsubsection{Intersección de círculos}
	% 		\inputminted[tabsize=2,breaklines,firstline=431,lastline=448,fontsize=\small]{c++}{geometry.cpp}
			
	% 		\subsubsection{Tangentes}
	% 		\inputminted[tabsize=2,breaklines,firstline=450,lastline=494,fontsize=\small]{c++}{geometry.cpp}
		
	% 	\subsection{Polígonos}
	% 		\subsubsection{Perímetro y área de un polígono}
	% 		\inputminted[tabsize=2,breaklines,firstline=174,lastline=190,fontsize=\small]{c++}{geometry.cpp}
			
	% 		\subsubsection{Envolvente convexa (convex hull) de un polígono}
	% 		\inputminted[tabsize=2,breaklines,firstline=192,lastline=211,fontsize=\small]{c++}{geometry.cpp}
			
	% 		\subsubsection{Verificar si un punto pertenece al perímetro de un polígono}
	% 		\inputminted[tabsize=2,breaklines,firstline=213,lastline=221,fontsize=\small]{c++}{geometry.cpp}
			
	% 		\subsubsection{Verificar si un punto pertenece a un polígono}
	% 		\inputminted[tabsize=2,breaklines,firstline=223,lastline=234,fontsize=\small]{c++}{geometry.cpp}
			
	% 		\subsubsection{Centroide de un polígono}
	% 		\inputminted[tabsize=2,breaklines,firstline=264,lastline=274,fontsize=\small]{c++}{geometry.cpp}
			
	% 		\subsubsection{Pares de puntos antipodales}
	% 		\inputminted[tabsize=2,breaklines,firstline=341,lastline=352,fontsize=\small]{c++}{geometry.cpp}
			
	% 		\subsubsection{Diámetro y ancho}
	% 		\inputminted[tabsize=2,breaklines,firstline=354,lastline=368,fontsize=\small]{c++}{geometry.cpp}
			
	% 		\subsubsection{Smallest enclosing rectangle}
	% 		\inputminted[tabsize=2,breaklines,firstline=370,lastline=389,fontsize=\small]{c++}{geometry.cpp}
		
	% 	\subsection{Par de puntos más cercanos}
	% 	\inputminted[tabsize=2,breaklines,firstline=236,lastline=262,fontsize=\small]{c++}{geometry.cpp}
		
	% 	\subsection{Vantage Point Tree (puntos más cercanos a cada punto)}
	% 	\inputminted[tabsize=2,breaklines,firstline=276,lastline=339,fontsize=\small]{c++}{geometry.cpp}
		
	% 	\subsection{Suma Minkowski}
	% 	\inputminted[tabsize=2,breaklines,firstline=496,lastline=517,fontsize=\small]{c++}{geometry.cpp}
		
	% \newpage
	% \section{Grafos}
	% 	\subsection{Estructura \texttt{disjointSet}}
	% 	\inputminted[tabsize=2,breaklines,firstline=8,lastline=37,fontsize=\small]{c++}{graph.cpp}
		
	% 	\subsection{Estructura \texttt{edge}}
	% 	\inputminted[tabsize=2,breaklines,firstline=39,lastline=57,fontsize=\small]{c++}{graph.cpp}
		
	% 	\subsection{Estructura \texttt{path}}
	% 	\inputminted[tabsize=2,breaklines,firstline=59,lastline=64,fontsize=\small]{c++}{graph.cpp}
		
	% 	\subsection{Estructura \texttt{graph}}
	% 	\inputminted[tabsize=2,breaklines,firstline=66,lastline=101,fontsize=\small]{c++}{graph.cpp}
		
	% 	\subsection{DFS genérica}
	% 	\inputminted[tabsize=2,breaklines,firstline=411,lastline=429,fontsize=\small]{c++}{graph.cpp}
		
	% 	\subsection{Dijkstra con reconstrucción del camino más corto con menos vértices}
	% 	\inputminted[tabsize=2,breaklines,firstline=103,lastline=129,fontsize=\small]{c++}{graph.cpp}
		
	% 	\subsection{Bellman Ford con reconstrucción del camino más corto con menos vértices}
	% 	\inputminted[tabsize=2,breaklines,firstline=131,lastline=165,fontsize=\small]{c++}{graph.cpp}
		
	% 	\subsection{Floyd}
	% 	\inputminted[tabsize=2,breaklines,firstline=167,lastline=175,fontsize=\small]{c++}{graph.cpp}
		
	% 	\subsection{Cerradura transitiva $O(V^3)$}
	% 	\inputminted[tabsize=2,breaklines,firstline=177,lastline=184,fontsize=\small]{c++}{graph.cpp}
		
	% 	\subsection{Cerradura transitiva $O(V^2)$}
	% 	\inputminted[tabsize=2,breaklines,firstline=186,lastline=200,fontsize=\small]{c++}{graph.cpp}
		
	% 	\subsection{Verificar si el grafo es bipartito}
	% 	\inputminted[tabsize=2,breaklines,firstline=202,lastline=224,fontsize=\small]{c++}{graph.cpp}
		
	% 	\subsection{Orden topológico}
	% 	\inputminted[tabsize=2,breaklines,firstline=226,lastline=252,fontsize=\small]{c++}{graph.cpp}

	% 	\subsection{Detectar ciclos}
	% 	\inputminted[tabsize=2,breaklines,firstline=254,lastline=274,fontsize=\small]{c++}{graph.cpp}
		
	% 	\subsection{Puentes y puntos de articulación}
	% 	\inputminted[tabsize=2,breaklines,firstline=276,lastline=304,fontsize=\small]{c++}{graph.cpp}
		
	% 	\subsection{Componentes fuertemente conexas}
	% 	\inputminted[tabsize=2,breaklines,firstline=306,lastline=335,fontsize=\small]{c++}{graph.cpp}
		
	% 	\subsection{Árbol mínimo de expansión (Kruskal)}
	% 	\inputminted[tabsize=2,breaklines,firstline=337,lastline=353,fontsize=\small]{c++}{graph.cpp}
		
	% 	\subsection{Máximo emparejamiento bipartito}
	% 	\inputminted[tabsize=2,breaklines,firstline=355,lastline=409,fontsize=\small]{c++}{graph.cpp}
		
	% 	\subsection{Circuito euleriano}
		
		
	% \newpage
	% \section{Árboles}		
	% 	\subsection{Estructura \texttt{tree}}
	% 	\inputminted[tabsize=2,breaklines,firstline=432,lastline=470,fontsize=\small]{c++}{graph.cpp}
		
	% 	\subsection{$k$-ésimo ancestro}
	% 	\inputminted[tabsize=2,breaklines,firstline=472,lastline=484,fontsize=\small]{c++}{graph.cpp}
		
	% 	\subsection{LCA}
	% 	\inputminted[tabsize=2,breaklines,firstline=486,lastline=505,fontsize=\small]{c++}{graph.cpp}
		
	% 	\subsection{Distancia entre dos nodos}
	% 	\inputminted[tabsize=2,breaklines,firstline=507,lastline=530,fontsize=\small]{c++}{graph.cpp}
		
	% 	\subsection{HLD}
		
		
	% 	\subsection{Link Cut}
		
		
	% \newpage
	% \section{Flujos}
	% 	\subsection{Estructura \texttt{flowEdge}}
	% 	\inputminted[tabsize=2,breaklines,firstline=4,lastline=17,fontsize=\small]{c++}{flow.cpp}
		
	% 	\subsection{Estructura \texttt{flowGraph}}
	% 	\inputminted[tabsize=2,breaklines,firstline=19,lastline=38,fontsize=\small]{c++}{flow.cpp}
		
	% 	\subsection{Algoritmo de Edmonds-Karp $O(VE^2)$}
	% 	\inputminted[tabsize=2,breaklines,firstline=82,lastline=108,fontsize=\small]{c++}{flow.cpp}
		
	% 	\subsection{Algoritmo de Dinic $O(V^2E)$}
	% 	\inputminted[tabsize=2,breaklines,firstline=40,lastline=80,fontsize=\small]{c++}{flow.cpp}
		
	% 	\subsection{Flujo máximo de costo mínimo}
	% 	\inputminted[tabsize=2,breaklines,firstline=110,lastline=145,fontsize=\small]{c++}{flow.cpp}
		
	% \newpage
	% \section{Estructuras de datos}
	% 	\subsection{Segment Tree}
	% 		\subsubsection{Point updates, range queries}
	% 		\inputminted[tabsize=2,breaklines,firstline=4,lastline=37,fontsize=\small]{c++}{queries.cpp}
			
	% 		\subsubsection{Dinamic with lazy propagation}
	% 		\inputminted[tabsize=2,breaklines,firstline=39,lastline=90,fontsize=\small]{c++}{queries.cpp}
		
	% 	\subsection{Fenwick Tree}
	% 	\inputminted[tabsize=2,breaklines,firstline=92,lastline=129,fontsize=\small]{c++}{queries.cpp}
		
	% 	\subsection{SQRT Decomposition}
	% 	\inputminted[tabsize=2,breaklines,firstline=131,lastline=209,fontsize=\small]{c++}{queries.cpp}
		
	% 	\subsection{AVL Tree}
	% 	\inputminted[tabsize=2,breaklines,firstline=211,lastline=414,fontsize=\small]{c++}{queries.cpp}
		
	% 	\subsection{Treap}
	% 	\inputminted[tabsize=2,breaklines,firstline=416,lastline=563,fontsize=\small]{c++}{queries.cpp}
		
	% 	\subsection{Ordered Set C++}
	% 	\inputminted[tabsize=2,breaklines,firstline=653,lastline=686,fontsize=\small]{c++}{queries.cpp}
		
	% 	\subsection{Splay Tree}
		
		
	% 	\subsection{Sparse table}
	% 	\inputminted[tabsize=2,breaklines,firstline=565,lastline=600,fontsize=\small]{c++}{queries.cpp}
		
	% 	\subsection{Wavelet Tree}
	% 	\inputminted[tabsize=2,breaklines,firstline=602,lastline=651,fontsize=\small]{c++}{queries.cpp}
		
	% 	\subsection{Red Black Tree}
		
		
	% \newpage
	% \section{Strings}
	% 	\subsection{Trie}
	% 	\inputminted[tabsize=2,breaklines,firstline=144,lastline=196,fontsize=\small]{c++}{strings.cpp}
		
	% 	\subsection{KMP}
	% 	\inputminted[tabsize=2,breaklines,firstline=4,lastline=39,fontsize=\small]{c++}{strings.cpp}
		
	% 	\subsection{Aho-Corasick}
	% 	\inputminted[tabsize=2,breaklines,firstline=41,lastline=142,fontsize=\small]{c++}{strings.cpp}
		
	% 	\subsection{Rabin-Karp}
		
		
	% 	\subsection{Suffix Array}
		
		
	% 	\subsection{Función Z}
		
	
	% \newpage
	% \section{Varios}
	% 	\subsection{Lectura y escritura de \texttt{\_\_int128}}
	% 	\inputminted[tabsize=2,breaklines,firstline=46,lastline=83,fontsize=\small]{c++}{misc.cpp}
		
	% 	\subsection{Longest Common Subsequence (LCS)}
	% 	\inputminted[tabsize=2,breaklines,firstline=21,lastline=33,fontsize=\small]{c++}{misc.cpp}
		
	% 	\subsection{Longest Increasing Subsequence (LIS)}
	% 	\inputminted[tabsize=2,breaklines,firstline=5,lastline=19,fontsize=\small]{c++}{misc.cpp}
		
	% 	\subsection{Levenshtein Distance}
	% 	\inputminted[tabsize=2,breaklines,firstline=145,lastline=156,fontsize=\small]{c++}{misc.cpp}
		
	% 	\subsection{Día de la semana}
	% 	\inputminted[tabsize=2,breaklines,firstline=35,lastline=44,fontsize=\small]{c++}{misc.cpp}
		
	% 	\subsection{2SAT}
	% 	\inputminted[tabsize=2,breaklines,firstline=85,lastline=128,fontsize=\small]{c++}{misc.cpp}
		
	% 	\subsection{Código Gray}
	% 	\inputminted[tabsize=2,breaklines,firstline=130,lastline=143,fontsize=\small]{c++}{misc.cpp}
\end{multicols}
\end{document}